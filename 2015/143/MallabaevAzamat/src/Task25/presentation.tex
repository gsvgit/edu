\documentclass{beamer}
\usepackage{beamerthemesplit}
\usepackage{wrapfig}
\usetheme{SPbGU}
\usepackage{pdfpages}
\usepackage{amsmath}
\usepackage{cmap}
\usepackage[T2A]{fontenc}
\usepackage[utf8]{inputenc}
\usepackage[english,russian]{babel}
\usepackage{indentfirst}
\usepackage{amsmath}
\usepackage{tikz}
\usepackage{multirow}
\usepackage[noend]{algpseudocode}
\usepackage{algorithm}
\usepackage{algorithmicx}
\usetikzlibrary{shapes,arrows}
\usepackage{fancyvrb}
\newtheorem{rutheorem}{Теорема}
\newtheorem{ruproof}{Доказательство}
\newtheorem{rudefinition}{Определение}
\newtheorem{rulemma}{Лемма}
\beamertemplatenavigationsymbolsempty

\title[]{Доказательство сортировки списков алгоритмом Хоара}
%\subtitle[]{}
% То, что в квадратных скобках, отображается в левом нижнем углу.
\institute[СПбГУ]{
Санкт-Петербургский государственный университет }

% То, что в квадратных скобках, отображается в левом нижнем углу.
\author[Маллабаев Азамат]{Маллабаев Азамат Нурмухамадович, 143 группа \\
  % У научного руководителя должна быть указана научная степень
  \and
    {\bfseries Научный руководитель:} ст.пр. С.В. Григорьев \\
  % Для курсовой не обязателен. Должна быть указана должность или ученая степень
  %\and
  %  {\bfseries Рецензент:} программист ООО ``Рога и копыта'' И.И. Иванов
  }

\date{2015}

\definecolor{orange}{RGB}{179,36,31}

\begin{document}
{
% Лого университета или организации, отображается в шапке титульного листа
\begin{frame}
  \begin{center}
  {\includegraphics[width=1cm]{pictures/SPbGU_Logo.png}}
  \end{center}
  \titlepage
\end{frame}
}

\begin{frame}[fragile]
  \transwipe[direction=90]
  \frametitle{Введение}
  \text{Разработан Чарльзом Хоаром во время работы в МГУ в 1960 году}
  \text{Эффективность порядка от O$(n log n)$ до O$(n^2)$}
\end{frame}

\begin{frame}
  \transwipe[direction=90]
  \frametitle{Алгоритм сортировки Хоара}
  \begin{enumerate}
    \item Взять опорный элемент (берем первый элемент списка)
    \item Разбить список на 3 части, где элементы
    \begin{enumerate}
      \item Меньше опорного
      \item Равны опорному
      \item Больше опорного
    \end{enumerate}
    \item Отсортировать отдельно 1ую и 3юю часть
    \item Объединить все 3 части
  \end{enumerate}
\end{frame}

\begin{frame}
  \transwipe[direction=90]
  \frametitle{Аналоги и инструменты}
  \text{Алгоритмы сортировки}
  \begin{itemize}
    \item Сортировка пузырьком
    \item Шейкерная сортировка
    \item Сортировка слиянием
  \end{itemize}
  \text{Инструмент реализации и доказательства - F*}
\end{frame}

% Обязательный слайд: четкая формулировка цели данной работы и постановка задачи
% Описание выносимых на защиту результатов, процесса или особенностей их достижения и т.д.
\begin{frame}
  \transwipe[direction=90]
  \frametitle{Постановка задачи}
  \textbf{Целью} работы является доказательство сортировки Хоара для списков

  \textbf{Задачи}
  \begin{enumerate}
    \item Реализовать алгоритм для сортировки списка чисел
    \item Провести доказательство корректности алгоритма
  \end{enumerate}
\end{frame}

\begin{frame}
  \transwipe[direction=90]
  \frametitle{Реализация алгоритма}
  \begin{rudefinition}
    module HoarSort\linebreak
    type List 'a =\linebreak
     - | Empty\linebreak
     - | Smth: hd:'a -> tl:List 'a -> List 'a\linebreak
    val part: int -> List int -> Tot (List int * List int * List int)\linebreak
    val merge: List int * List int * List int -> Tot (List int)\linebreak
    val qsort: List int -> Tot (List int)\linebreak
    let rec qsort lst =\linebreak
     - match lst with\linebreak
     - | Empty -> Empty\linebreak
     - | Smth h t ->\linebreak
     -  - let al, bl, sl = part h lst\linebreak
     -  - merge (qsort al) bl (qsort sl)\linebreak
  \end{rudefinition}
\end{frame}

\begin{frame}
  \transwipe[direction=90]
  \frametitle{Доказательство корректности алгоритма}
  \begin{ruproof}
    module Proof\linebreak
    open HoarSort\linebreak
    val sorted: MyList int -> Tot bool\linebreak
    let sorted lst =\linebreak
     - match lst with\linebreak
     - | Empty -> true\linebreak
     - | Smth a Empty -> true\linebreak
     - | Smth a (Smth b t) ->\linebreak
     -  - if a > b\linebreak
     -  - then false\linebreak
     -  - else sorted (Smth b t)\linebreak
    val sortedHoarSortResult: MyList 'a -> Tot unit\linebreak
    let sortedHoarSortResult lst =\linebreak
     - assert(sorted (qsort lst)) \linebreak
  \end{ruproof}
\end{frame}

\begin{frame}
  \transwipe[direction=90]
  \frametitle{Результаты}
  \text{Выполнены задачи}
  \begin{itemize}
    \item Реализована сортировка Хоара для списков чисел
    \item Проведено доказательство корректности алгоритма
  \end{itemize}
  \text{В ходе работы выявлены проблемы инструмента доказательства}
  \begin{itemize}
    \item Неустоявшийся синтаксис
    \item Недоработанный верификатор
    \item Неустоявшийся интерфейс
  \end{itemize}
\end{frame}

\end{document} 