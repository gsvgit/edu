\documentclass{beamer}
\usepackage{beamerthemesplit}
\usepackage{wrapfig}
\usetheme{SPbGU}
\usepackage{pdfpages}
\usepackage{amsmath}
\usepackage{cmap} 
\usepackage[T2A]{fontenc} 
\usepackage[utf8]{inputenc}
\usepackage[english,russian]{babel}
\usepackage{indentfirst}
\usepackage{amsmath}
\usepackage{tikz}
\usepackage{multirow}
\usepackage[noend]{algpseudocode}
\usepackage{algorithm}
\usepackage{algorithmicx}
\usetikzlibrary{shapes,arrows}
\usepackage{fancyvrb}
\newtheorem{rutheorem}{Теорема}
\newtheorem{ruproof}{Доказательство}
\newtheorem{rudefinition}{Определение}
\newtheorem{rulemma}{Лемма}
\beamertemplatenavigationsymbolsempty

\title[]{Сортировка списков алгоритмом Хоара}
%\subtitle[]{}
% То, что в квадратных скобках, отображается в левом нижнем углу. 
\institute[СПбГУ]{
Санкт-Петербургский государственный университет \\
Кафедра системного программирования }

% То, что в квадратных скобках, отображается в левом нижнем углу.
\author[Маллабаев Азамат]{Маллабаев Азамат Нурмухамадович, 143 группа \\
  % У научного руководителя должна быть указана научная степень
  \and  
    {\bfseries Научный руководитель:} ст.пр. С.В. Григорьев \\ 
  % Для курсовой не обязателен. Должна быть указана должность или ученая степень
  %\and
  %  {\bfseries Рецензент:} программист ООО ``Рога и копыта'' И.И. Иванов
  }

\date{ноябрь 2015г.}

\definecolor{orange}{RGB}{179,36,31}

\begin{document}
{
% Лого университета или организации, отображается в шапке титульного листа
\begin{frame}
  \begin{center}
  {\includegraphics[width=1cm]{pictures/SPbGU_Logo.png}}
  \end{center}
  \titlepage
\end{frame}
}

\begin{frame}[fragile]
  \transwipe[direction=90]
  \frametitle{Введение}
  \begin{itemize}
    \item Алгоритмы сортировки
    \item Конструируемые структуры данных
  \end{itemize}
\end{frame}
            
\begin{frame}
  \transwipe[direction=90]
  \frametitle{Аналоги и инструменты}
  \begin{itemize}
    \item Похожие алгоритмы
    \begin{itemize}
      \item Алгоритм Хоара - наиболее эффективен для массивов
      \item Алгоритм сортировки слиянием - рекурсивен
      \item Алгоритм плавной сортировки и сортировки кучей - требует создания сортирующего дерева
    \end{itemize}
    \item Инструмент реализации          
    \begin{itemize}                                              
      \item F* - для реализации алгоритма и проверки его корректности         
    \end{itemize}
  \end{itemize}
\end{frame}

% Обязательный слайд: четкая формулировка цели данной работы и постановка задачи
% Описание выносимых на защиту результатов, процесса или особенностей их достижения и т.д.
\begin{frame}
  \transwipe[direction=90]
  \frametitle{Постановка задачи}
  \textbf{Целью} работы является разработка алгоритма Хоара, применимого для списков

  \textbf{Задачи:}
  \begin{itemize}
    \item Разработать алгоритм, реализующий сортировку Хоара для списков
    \item Проверить корректность и реализовать предложенный алгоритм
    \item Провести проверку
  \end{itemize}
\end{frame}

\begin{frame}
  \transwipe[direction=90]
  \frametitle{Проверка и реализация алгоритма}
  \begin{itemize}
    \item Проверка выходных данных с помощью зависимых типов
    \item Реализованы следующие составляющие алгоритм функции:
    \begin{itemize}
      \item Конструкторы типа, реализующего список
      \item Выбор элементов списка по шаблону
      \item Выбор опорного элемента
      \item Слияние списков
    \end{itemize} 
  \end{itemize}
\end{frame}

\begin{frame}
  \transwipe[direction=90]
  \frametitle{Проверка реализации}
  \begin{itemize}
    \item Тестирование проводили на следующих списках:
    \begin{itemize}
      \item Из одинаковых и из полностью отсортированных элементов - стабильности
      \item Из отсортированных в обратном порядке - работа
      \item Из частично отсорированных данных - сложная работа
      \item Из случайных данных - проблемные данные
    \end{itemize}             
  \end{itemize}
\end{frame}
                                                           
\begin{frame}
  \transwipe[direction=90]
  \frametitle{Результаты}
  \begin{itemize}
    \item Разработан алгоритм, реализующий сортировку Хоара для списков
    \item Проверена корректность и реализован предложенный алгоритм
    \item Проведена проверка
  \end{itemize}
\end{frame}

\end{document}